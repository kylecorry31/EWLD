\documentclass{book}
\usepackage[letterpaper,margin=1.0in]{geometry}
\usepackage[utf8]{inputenc}

\title{Exploring the World of Lucid Dreaming}
\author{Kyle Corry}
\begin{document}
\maketitle
\setcounter{tocdepth}{1}

\tableofcontents

\chapter*{Overview and Attribution}
\addcontentsline{toc}{chapter}{Overview and Attribution}
The following work is a summary of the book \textit{Exploring the World of Lucid Dreaming} by LeBerge \cite{leberge}. All ideas and techniques presented here, unless otherwise stated, are from this book. This is a bare bones summary of the book - it does not include examples, stories, some techniques, or detailed descriptions as were presented in the book. To view those, you should buy the actual book, which is fairly cheap on Amazon and worth a read if you are actually serious about lucid dreaming. This summary is meant as a way to jog your memory once you read the book, so you do not need to read it over again to find information you are looking for in a succinct format. This summary covers 313 pages into just XX pages, so obviously a lot of details are missing in this summary (buy the book to find out more!).

\chapter{The World of Lucid Dreaming}
\section{The Wonders of Lucid Dreaming}
Most people don't realize that they are dreaming when weird things happen around them that defy all explanations. When you do realize that the explanation for something is that you are dreaming, you can become lucid (aka conscious that you are in a dream).

Upon becoming lucid, some dreamers have the opportunity to change and create the world around them as well as experience things that would be unlikely or impossible in real life. Some use lucid dreaming for fun, problem solving, personal growth, or even health.

Lucid dreaming has been known for ages, but it wasn't until recently that it became scientifically proven. Lucid dreaming is like any other skill, it can be learned and improved upon. Also, some people are naturally better at lucid dreaming.

Everyone can (and should) experience lucid dreaming, except for those who have trouble distinguishing the real world from that of their dream world. While many people believe that lucid dreaming could make you lose touch with reality, it has been shown that it actually makes you more aware and can make you better overall as you know yourself better.

\section{Life is Short}
Lucid dreaming allows you to be aware of the world around (well inside) you while you are sleeping. This expands the amount of time that you are conscious and allows you to perceive that you live longer. With lucid dreaming, you also wake up feeling positive and energized as most dreams are amazing to be in. You can also solve some of your problems in your dreams, leading to more time spent on other things in the waking world.

\paragraph{Exercise - your present state of consciousness:} Go through each of your senses and experience what is around you. Take a few minutes for each your sight, hearing, touch, taste, and smell. Then focus on your breathing - feel the air moving in and out of your body. Take deep breaths. Become aware of your emotions and how you are currently feeling. Remember the difference between your different emotions. Now become aware of your thoughts and what you have been thinking about while doing this exercise. How real do your thoughts feel? Become aware of yourself and who you are. Become aware that you exist. Finally, become aware that you are aware.

\section{Lucid Dreaming and Waking Life}
The major difference between lucid dreaming and waking life is the fact that the world that you see in your dreams is disconnected with that of the real world. Things can happen in your dreams that can't in the real world. Also objects in your dreams are not as stable as those in the real world because they don't have anything to ground themselves to other than your thoughts. For example, the same words on this page will be here if you look away and check again. The dream world is not stable like this, double checking on something could lead to two different apparitions - it is fluid and changeable. Lucid dreams can tend to be even more unstable than regular dreams, because you are both subconsciously and consciously changing your world. Lucid dreams also come with the sense of more freedom. There are no external actions, no laws (society or physics). Basically your dreams are limitless and you can use them to add to your real life.


\chapter{Preparation for Learning Lucid Dreaming}
\section{Learning How to Learn}
Some people start lucid dreaming after they learn about it, but that is probably due to beginner's luck. For the majority of people though, it is a skill that needs to be learned. It is recommended to start with a dream journal. A dream journal will help to improve your dream recall, and you will discover what your dreams are like. You can then use your dream journal to find \textbf{dreamsigns} which are oddities in your dream that appear often enough to become reliable ways to discover that you are dreaming.

\section{Sleeping Brain, Dreaming Mind}
In daily life, your brain takes what your senses are receiving and turns it into what you experience as consciousness. This means that everything you experience is created by your brain on some level.

\subsection{The mind in sleep}
When you are awake and active (doing something), your brain will shift its processing toward the external world. As you become more inactive, it will shift the balance a bit to allow you to be aware of your surroundings, yet focusing on the internal world in your mind (generally described as 'autopilot' or 'day dreaming'). At this point, your brain is still generating a model of the outside world, albeit reduced. When you are sleeping, your brain becomes much less aware of outside stimuli and your external model is at a minimum (just enough to awake you if any signs of danger appear). During this stage, your brain creates its model of the world based on your inner thoughts - a world not based on the external world.

Sleep though, is composed of several different parts. They can be thought of as a quiet phase and an active phase. The quiet phase is what is known as deep sleep. Your mind does very little during this stage. The active stage on the other hand involves your mind being very active - this is known as REM sleep (rapid eye movement, because your eyes dart around as if you were awake). Breathing during this stage is also much quicker than the slow, deep breaths from the deep sleep phase. Dreams happen during REM sleep.

The body is also paralyzed during REM sleep, which you may find out if you awaken before your body releases the chemicals to stop the paralysis. This is known as sleep paralysis. Sleep paralysis can be used to induce lucid dreams.

\subsection{The sleeper's night journey}
Quiet sleep is broken into three different stages. The first stage is a transition between drowsy wakefulness and light sleep. This stage involves \textbf{hypnagogic imagery} - small, brief dreamlets. Stage 2 is characterized by little mental activity. This stage leads the way to stage 3, which is called 'delta sleep' - essentially deep sleep. Little to no dreams occur during this stage. You typically stay in quiet sleep for around 1 to 1 and a half hours.

The next stage, is the active phase of sleeping - REM sleep. You typically stay in REM for a few minutes, then wake up. Normally you can remember your dream. You then fall asleep quickly and enter stage 2 of quiet sleep before going back into another REM. This one is a little longer than the last REM sleep. This cycle repeats approximately every 90 minutes, with REM becoming a bit longer as the night goes on. As REM sleep becomes longer, the intervals between them decrease to about 20 or 30 minutes (from 90). This typically can happen between 10 and 15 times in a single night.

\subsection{Communiqu\'{e} from the dream world}
These stages have been proven scientifically, and lucid dreaming was confirmed because participants were told to do actions in their dreams, which could be detected in the waking world by scientists. This was done because your eyes look where you are looking in your dreams. They were told to move their eyes left to right a certain number of times when they became lucid - this happened during REM sleep.

\section{Social Values and Lucid Dreaming}
It is hard to talk to people about what you do in your dream world because most people can't do it themselves or wouldn't have interest in doing it in the first place. Science has helped to reduce this barrier, but it is still there.

Lucid dreams can have a significant and valuable effect on your life, and it can be worth sharing. It may be necessary to find a group to talk with and share experiences and techniques.

\section{Getting to Know Your Dreams}
\subsection{How to recall your dreams}
Recalling your dreams is dependent on your memory but is necessary to continue learning how to lucid dream. Excellent dream recall is a requirement to lucid dreaming on command. It is entirely possible that you have had lucid dreams in the past, but just didn't remember them because you were not practicing dream recall. Dream recall is also necessary for you to recognize that a dream is in fact a dream - this goes hand in hand with keeping a dream journal.

\textbf{You need to recall your dreams so you can figure out what is dreamlike about them in order to realize you are dreaming in the future.}

Before attempting to use lucid dream induction techniques, it is very important that you can recall at least one dream each night.

To start with dream recall, it is good to be well rested - set enough time aside to sleep each night. This will help because it will be less miserable to take the time each night to record your dreams. Dream periods also increase the longer you sleep, so sleeping more will increase your changes of a lucid dream.

You could also set an alarm clock at 90 minute intervals if you have trouble awakening from your dreams due to being a deep sleeper. Aim to wake up after the later REM periods which are at 4 and a half, 6, or 7 and half hours after going to sleep.

It is also very important to be motivated to recall your dreams. It is typically enough to remind yourself to record your dreams just before bed. Keep a dream journal beside your bed, so when you wake up you can take quick notes about what you just dreamed about - take special care to list anything you found unusual as that will be among the most valuable information from your dreams. While you are recording your dreams, you will likely began to remember more.

Do not move from the position you woke up in without starting to remember what you just dreamed about. When you have an idea, start writing it down including your emotions. Don't think of anything else at this point other than what you just dreamed about. If you can't remember your dream, record your emotions and thoughts at the moment - they may be related to your dream.

Try to cling to any clues as to what you were dreaming, and try to rebuild the story. When you recall a dream, try to think of what the dream was before that until you run out of thoughts.

This skill won't take long to build, just a few nights after starting you should begin to remember much more dreams.

Once you can recall most of your dreams, you may already be experiencing lucid dreams.

\subsection{Keeping a dream journal}
Use a small notebook or diary which you place next to your bed. Keep a pen or pencil handy as well. Upon waking up from a dream (no matter the time of night), either write down your whole dream or take notes which will help you to remember what you dreamt about.

You can also use pictures in your dream journal to help you describe stuff: remember your dream journal will only be read by you, so it can't hurt to add stuff that only you will get. Nobody is going to judge your art!

At the start of each entry, put the date and record your dream. and leave a few blank pages. Leave some space under each dream for some later exercises like identifying dream symbols. If you manage to recall the whole dream, give it a catchy title as it can help you remember the dream quickly when looking over your journal.

Once you accumulate some dreams in the journal, look over them and ask yourself questions about them and what the might have in common/represent.

By doing this, you will get a better understanding of your mind and how it symbolizes things in the real world. Reading over your journals will also help you identify the elements of which are dreamlike so you will be able to realize them when in a regular dream to turn it into a lucid one.

\section{Dreamsigns: Doors to Lucidity}
Things that appear in multiple dreams which are out of place, or impossible in the real world can be considered dreamsigns. Nearly every dream has dreamsigns, and they differ for every person. Once you have a stock pile of dreams in your dream journal, you can use them to identify your personal dreamsigns. Use that to learn the most frequent and obvious characteristics which identify it as a dream. Most people don't recognize their dreamsigns because they are used to seeing them in their dreams and their dreaming brain comes up with an explanation of why it is happening. Dreamsigns typically fall into several different categories:
\begin{itemize}
  \item \textbf{Inner awareness} - These refer to the things that you perceive as happening within yourselves such as thoughts or feelings.
  \item \textbf{Action} - This includes the actions and motions of everything in the dream world. This includes the dream ego (who you perceive yourself as), other dream characters, and objects.
  \item \textbf{Form} - This refers to the shapes of things, people, and places. These objects can also transform in dreams to be deformed.
  \item \textbf{Context} - This is the combination of elements (including people, places, actions, or things) which is deemed odd, but each item seems normal. This also includes events (finding yourself in a place you are unlikely to be), meeting other characters in unusual places, finding out of place objects, or playing a different, unusual role.
\end{itemize}

\subsection{The Dreamsign Inventory \cite{leberge}}
\paragraph{Inner awareness} If you encounter a weird thought, very strong emotion (maybe one you don't feel much), feel an unusual sensation, or have altered perceptions then you might have an inner awareness dreamsign. If a thought can affect the world around you in some mysterious way, it is a dreamsign. If you have an inappropriate or overwhelming emotion that is also a dreamsign. Dreamsign sensations can include feeling paralyzed, a dark presence, leaving your body, or any weird physical feelings. Finally perception dreamsigns can make the world around you seem oddly clear or fuzzy and you may be able to hear stuff you shouldn't be able to.

\paragraph{Action} If something in your dream (including you, others, or objects) do something weird or impossible - also the action must be performed in the dream environment, and not in your mind.

\paragraph{Form} Your shape, a character's shape, or a dream object's shape is odd, deformed, or changes form. Also unusual aspects of people or things may also be considered form dreamsigns. Also a place that you are in may be altered.


\paragraph{Context} This is when the place or situation in your dream is strange. You may find yourself somewhere in which you are not likely to be in waking life, or you are in a strange social situation. Also you or a dream character could be in a role that you are not in in real life. Objects or people may also be out of place or the time may be off.

\paragraph{Exercise - Cataloging your dreamsigns:} The first step to cataloging your dreamsigns is to keep a dream journal. You should be recording all of the dreams that you remember and your thoughts as to what your dream could have been about if you didn't remember. Once you have at least a dozen dreams, you can move on. The second step is to mark your dreamsigns. Underline the things that you believe are dreamsigns, or make a list of them after each dream. For each dreamsign you marked, classify them based on the Dreamsign Inventory above. Tally up how many dreamsigns you have in each category and then rank the categories based on frequency. Your target dreamsign will be the one that you see most often. Pick your favorite category in the case of a tie. Finally, look for your dreamsigns while you are awake. Look for events that would fall under your category, this will help you notice when something is out of place in your dreams.

\section{Goal Setting for Success}
Like most other activities, lucid dreaming can be improved upon through goal setting. While starting out, set reasonable goals for yourself such as I want to recall my dreams tonight or later on I want to become lucid tonight (or do X while lucid). This will encourage your brain to succeed, and you will find that it will help your learn this skill.

\paragraph{Exercise - Goal setting for success:} Set specific goals which are personal and dependent on your current ability. This could be something as simple as I want to remember at least one dream this week or something more complex such as I want to do X while lucid tonight. Setting numeric goals will also help to judge your progress as you can compare the number that you completed to the number that you wanted to complete to see how far behind or ahead you are. Keep the goals realistic, but it is also good to make them difficult. You should also set short range and long range goals - short range could be achieve X in a week while long range could be achieve Y in a few months or a year. When you have reached your goal, record your progress and then set a new, more difficult goal. If you did not reach your goal, make a simpler goal that you can achieve first and then try again. Don't give up!

\section{How to Schedule Your Efforts for Best Results}
Most lucid dreams happen after dawn, in the later hours of your sleep cycle. The probability of having a lucid dream also seems to increase as the number of REM periods increase, meaning the more you sleep, the more likely you are to have a lucid dream. So if you want to increase the occurrences of lucid dreams, you could increase the amount you sleep. You can also redistribute your sleeping hours to make more of them after dawn. You can do this by waking up much earlier than normal, doing whatever you have to do and then going back to bed for a few hours.

\paragraph{Exercise - Scheduling time for lucid dreaming:} Set an alarm before going to bed to wake you up a few hours earlier than you normally do. Go to bed at your normal time though. When your alarm sounds, get out of bed without hesitation and do whatever you would normally do in the morning. Stay awake for a few hours (until an half an hour before the time you would normally be awake at), then go back to bed. Spend about half an hour thinking about what you want to accomplish in a lucid dream. Create a dream world in your mind. Now you can practice an induction technique and give yourself at least 2 hours of sleep.

\section{Final Preparations: Learning to Relax Deeply}
Before you can practice lucid dream induction techniques, you must be able to put yourself into a state where your body is relaxed but your mind is alert. Lucid dreaming requires concentration.

\paragraph{Exercise - Progressive relaxation:} For this relaxation technique, lie down and get comfortable. Pay attention to your breathing and allow it to deepen. Let tension escape with each breath. For each muscle group in your body, tense it and then relax it. Start with your dominant arm and work around your body. Tense each group for about 5 to 10 seconds and pay attention to the tension. Relax the muscles and recognize the difference. Breathe deeply and slowly. Repeat this process a second time for each muscle group before moving on to the next one. The muscle group sequence should be: wrists, forearms, upper arms, forehead, jaw, neck, shoulders, abdomen, back, buttocks, legs, and finally feet. Pause between each group and take a deep breath. Once your have completed all of the muscle groups, release any remaining tension.

\bigskip\noindent There is also another relaxation technique called 61 point relaxation, but I will not summarize it here.



\chapter{Waking Up in the Dream World}
Before continuing with this section, be sure that you can remember at least one dream per night. You should also have at least a dozen dreams recorded in a journal. From those dreams, you should have extracted dream signs. If all of that is done, you are ready to learn techniques which will help you have your first lucid dream.

\section{Lucid Dreaming is Easier Than You May Think}
Lucid dreaming isn't actually hard to do, many times it is difficult because you impose mental blocks by thinking that it is difficult. By believing that lucid dreaming is difficult to achieve, it actually becomes hard to do. Basically if you practice the techniques presented in this book, you are almost guaranteed to succeed. The amount of time it will take before you are able to lucid dream is entirely dependent on you. Factors which affect it include dream recall, motivation, the amount you practice, and a bit of a natural talent to be able to lucid dream. If you are having trouble lucid dreaming, keep on practicing and believing that it is easy to lucid dream and eventually you will be able to as well.

\subsection{Find the technique that works best for you}
The techniques presented throughout the next sections will help most people lucid dream, but you may have to tweak them based on your lifestyle and preferences. You may find that the best technique is a combination of the ones presented to you in the book rather than any of them individually.

\section{Critical State Testing}
\subsection{Building a bridge between the two worlds}
Throughout the day, stop and ask yourself if you are awake or dreaming. Be serious when you ask this. If you don't ask yourself this during the day, you shouldn't expect yourself to ask this when you are dreaming, and thus won't realize that you are dreaming. By making this into a habit in the waking world, you will likely do this at some point in your dreams, and then become lucid. Ask yourself this question between 5 to 10 times a day and every time you are in a situation which could potentially be dreamlike. Use your dreamsigns to help you discover what could be dreamlike and then do a reality check (as I call it) to see if you are dreaming.

Here are some tips:
\begin{itemize}
  \item Plan to test your state several times during the day.
  \item Test your state anytime you see or experience something weird (such as something that might be your dreamsign)
  \item Don't just answer the question "are you awake?" with yes, think about it - do some reality checks and try to think about the events of the last few minutes
\end{itemize}

\section{Tips on state testing}
Many times in dreams you see something that really shouldn't exist, but many times ignore it or tell yourself that it has an explanation and you are awake. Doing proper reality checks will help to prevent situations like this from occurring. Also, asking someone around you is not a good method - if you are awake, they will think you are crazy and if you are sleeping they will act like you are crazy.

Here are some quick reality checks that you can do (some differ from the book and are based off of the ones presented on r/LucidDreaming):
\begin{itemize}
  \item Jump into the air and see if you stay airborne or hover for longer than you should
  \item Pinch your nose closed and close your mouth. Attempt to breathe in - in a dream you can
  \item Try to push your finger through your palm, in a dream you may be able to do this
  \item Count your fingers, the fingers of your dream ego may differ in number
  \item Read some text, look away and read it again - it should be the same in waking life
  \item Look at your digital watch (supposedly it won't work with analog), see if the time changes unusually or if the interface is scrambled
\end{itemize}

Remember, anytime you really suspect yourself of dreaming, you probably are (do a reality check first though!)

\section{Intention Techniques}
Intention techniques help you maintain a state of consciousness from the waking world into that of your dream world, allowing you to lucid dream.

\paragraph{Power of resolution technique} This technique involves you believing that you will maintain a continuous steam of consciousness - meaning that you will be conscious all day and night. During the day, continue to think that everything you experience is being fabricated in your mind. Resolve to realize their true nature. At night, while trying to sleep resolve that you will realize that you are dreaming.

\paragraph{Intention technique} Resolve to recognize that you are dreaming if you awaken during the later part of your sleep period. Visualize yourself recognizing that you are in a dream. Incorporate your dreamsigns into that imagery. Then imagine yourself carrying out an action that you would like to do in a lucid dream after you recognize it is a dream.

\subsection{Mnemonic Induction of Lucid Dreams (MILD)}
MILD allows the dreamer to remember more dreams (improve dream recall) as well as have more lucid dreams, just by convincing yourself that you will remember dreams/have lucid dreams. This technique uses your \textbf{prospective memory}, which is strongly affected by motivation. This type of memory is how you remember to do stuff that you are not reminded to do. This works with the association that your brain makes with signs that help you to remember what your goal was.

\paragraph{MILD prerequisites} In order to practice MILD, you must be able to remember to do stuff during the day without reminders. If this is an issue, or you do find yourself using external reminders for most tasks do the following exercise.

\paragraph{Exercise - Prospective memory training:} This exercise will take a week to complete, Create a list of 4 different targets for each day of the week. An example would be: next time I see an animal. Look at the targets for the day in the morning and memorize them. During the day, look for your targets. On the first time each of your targets are met, you will perform a reality check / state check. Keep track of how many targets you hit or miss. If you think you hit a target, but then later remember that you saw the target earlier in the day, make note of that - you have missed the target for that day. Keep track of this every day of the week, and look for improvements. Continue this exercise for at least a week, or until you can hit all or most of your targets.

\paragraph{MILD Technique} Before going to bed, resolve to recall your dreams every time you wake up. When you wake up from a dream period, recall your dream in as much detail as possible - don't let yourself fall asleep until you have your dream recalled. When you are ready to fall back asleep, resolve to remember to recognize that you are dreaming. Focus your thoughts on this idea alone. At the same time as that, imagine that you are back in the dream that you just had, but this time you notice that you are dreaming. Find a dreamsign that you can recognize - tell yourself that you are dreaming, and continue out your dream being lucid. Repeat the focusing of your thoughts and the imagination of being lucid until your intention is set and you are ready to fall asleep. Don't let yourself think of anything else while falling asleep. Make sure lucid dreaming is the last thing on your mind. If you follow this, hopefully you will recognize that you are dreaming.


\chapter{Falling Asleep Consciously}
While the previous section involved taking an idea from the waking world into your dreams, to allow you to recognized that you are dreaming and become lucid, this section will show you how to become lucid because you never lost consciousness to begin with. Dreams that you enter while never losing consciousness are called wake-initiated lucid dreams (WILDs) while the ones in the previous chapter are called dream-initiated lucid dreams (DILDs).

\section{Wake Induced Lucid Dreams (WILDs)}
The idea of falling asleep consciously involves you focusing on something while you fall asleep which could include the hypnagogic imagery, visualizations, your breath / heartbeat, or senses. The goal is to let your body fall asleep, but keep your mind active enough that you will find yourself in REM sleep while still conscious.

\section{Attention to Hypnagogic Imagery}
When you close your eyes to sleep, you may see flashes of light and small dots. Focus your attention on those lights. Gradually they will begin to transform into more complex shapes, and eventually faces or objects and finally a dream scene. These images are much more likely to occur if you wake up in the night or early morning hours directly after a REM sleep. It is very difficult to have them occur while you are initially falling asleep.

\paragraph{Hypnagogic imagery technique} While you are lying in bed, relax your body completely, either using the relaxation technique presented earlier, or if you just woke up from REM sleep, you may be relaxed already. Let your mind become calm as well. Next focus your attention on the images that appear in front of your closed eyes. Observe them passively, so without attaching to any specific imagery. Just let them do their thing, don't attempt to interact, just watch. As the imagery forms into a dream scene, let yourself be pulled into the dream, don't forcefully enter the dream. In my experience, you will see the dream scene and feel like you are being pulled down a tunnel, sometimes with bright light, and before you know it you will be in a dream. Don't become too passive, as you may forget that you are in a dream!

\section{Attention to Visualization}
It is possible to induce a lucid dream by focusing on the visualization of a symbol while you are relaxed, much like the hypnagogic imagery technique. The only difference between the two, is that instead of acting as a passive observer, you are forcing the visualizations and being more of an active observer.

\paragraph{White dot technique} Before going to bed, resolve that you are going to recognize that you are dreaming. Once you wake up toward the later hours of the night / early morning, you must relax and focus on taking deep breaths. Focus on your breathing. Resolve to recognize that you are dreaming. Concentrate on a single white dot that is floating between your eyes. Focus on the dot until you are dreaming. 


\paragraph{Other techniques} There were several other techniques covered in the book, I will not cover them here as they seem to be several variations of the white dot technique. If you are interested in learning them, purchase the book.

\section{Attention to Other Mental Tasks}
While falling asleep, focus on other cognitive tasks which will keep your mind busy while still allowing your body to fall asleep. 

\paragraph{Count your self to sleep technique}
Once completely relaxed, start counting yourself to sleep. Say each number slowly in your head, followed by the statement "I am dreaming". Repeat this until you realize you are dreaming. You may also find it more convenient to restart at 1 when reaching a certain number, like 100. Do a mental state check after a certain count. 

\section{Attention on Body or Self}
While falling asleep, you can focus on your body and how it feels. After a while you will notice strange things happening, such as twitches, vibrations, or overall paralysis. Once in a state of paralysis, you can do several different things.

\paragraph{A note on sleep paralysis} Sleep paralysis is a normally occurring phenomena, scary things may seem to be happening such as demons being around you or the feeling that you can't breathe. Ignore these things, as they are produced by your half-sleeping brain and aren't really happening. Sleep paralysis can occur before or after entering REM sleep.

\paragraph{Visualizing a dream scene in sleep paralysis} Once in sleep paralysis, you can visualize a scene which you would like to be. Just close your eyes and picture where you want to go. You may see a bright light and feel as if you are being pulled into the scene. Allow this to happen, and you will be lucid. You can do this anytime you find yourself in sleep paralysis. 

\bigskip\noindent
The book covers a few more techniques that you can do when entering sleep paralysis involving the brain's body image. It talks about how you can have 2 bodies (1 in the real world, the other in the dream world), 1 body (just an image of a body which can be in either the dream or real world), or no body (just a dot). I will not go into any in detail, again buy the book to learn more. 

\section{Where Do You Go From Here?}
Try all of the different techniques covered in the previous sections before moving on, and try to practice the one which seems to work the best for you. If you are unable to lucid dream at this point (even for a few seconds), keep practicing the stuff covered earlier in this book until something works. In the next sections, you will learn what to do once you are lucid. 


\chapter{The Building of Dreams}
This section discusses how dreams are actually created and some of the science and psychology behind it. Essentially, dreams are our brains making a simulation of the world based on internal information without actual sensory input.

\section{The Construction of Perception}
Perception is a complex topic, but it seems to have a basis in both expectation and motivation.

\subsection{Expectation and perception}
Perception seems to be based on expectation. If you expect something will happen based on past experiences, you perception may be slightly altered to favor that outcome. Some outcomes that don't line up with what you expect may even be ignored in the real life.

\subsection{Motivation and perception}
Motivation also determines what we perceive. If you are missing a basic need, you are more likely to perceive that need in places where it is not. The example given demonstrates that if you are hungry, you are more likely to see/smell food than you would if you are full. Emotions also motivate what you perceive. 

\section{Schemas: Building Blocks of the Mind}
Schemas are an idea that the brain has some type of pattern matching capabilities which consist of networks of neurons. Schemas can be learned, forgotten, adapted, combined, generalized, or specialized. Schemas can also organize different experiences by grouping features or attributes together for a specific situation, person, or object allowing you to perceive it as a whole.

\subsection{Schemas for everything}
Schemas are used for everything in your brain. They can identify certain social situations, allowing you to act appropriately, or they can identify a scene, object, or action, allowing you to know where you are and what is going on.

\subsection{Schema activation}
There is thought to be three different activation levels of the schemas: conscious, preconscious, and unconscious. Schemas that are inactive or are receiving little activity can be considered unconscious. If the schema reaches a certain threshold, and is highly activated it falls into the conscious mind. You are aware that the schema is present. If it is neither so inactive to fall into the unconscious mind or active enough to become conscious, it can be considered to be preconscious. Being preconscious allows the schema to be easily activated when looking at patterns. 

\section{A Model of Dreaming}
\subsection{The building of dreams}
As the brain enters REM sleep, it stops receiving most input from the senses. This means that when you are dreaming, you are experiencing things that are not present in the physical environment. You brain allows for weird things to happen, because it can't double check with the senses to make sure it doesn't exist. 

\subsection{What we are likely to dream about}
Dreams are determined by which schemas are activated above the conscious threshold. Going to sleep thinking about certain things, may allow them to make an appearance in your dreams. All dream worlds resemble places that we have actually seen or been to, thus expectation plays a big role in dreams. Characters in dreams may resemble what we expect and actions that happen are most likely due to expectation. 

\subsection{Why dreams are meaningful}
Dreams can tell you a lot about yourself: your personal interests, concerns, experiences, preoccupations, fears, and personality. Looking through your dream journal may reveal things you already know about yourself (such as your interests or fears), but they may also reveal things that you did not realize, allowing you to be more conscious of them in the real world.

\section{Mental Constraints on Dreaming} 
Be careful of having strong assumptions, as your schemas can make the world fit to that assumption, blocking your brain from learning new things. This can also affect dreams, as you can make assumptions and cause things to become impossible. If you expect that something can't happen in your dreams, it won't happen. You can use this to prevent nightmares, but you can also accidentally use this to cause nightmares or limit your abilities in a dream.



\chapter{Principles and Practice of Lucid Dreaming}


\begin{thebibliography}{9}

\bibitem{leberge}
  Stephen LeBerge and Howard Rheingold,
  \textit{Exploring the World of Lucid Dreaming},
  Ballantine Books, New York,
  1990.

\end{thebibliography}


\end{document}
